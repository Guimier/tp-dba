\subsection{Gestion des utilisateurs}

\subsubsection{Création de Bob}

\begin{verbatim}
CREATE USER bob IDENTIFIED BY along DEFAULT TABLESPACE users QUOTA 1 M ON users;
GRANT connect TO bob;
GRANT CREATE TABLE TO bob;

\end{verbatim}

\subsubsection{Création de Kay et compie de \texttt{emp}}

\paragraph{Création de Kay}
\begin{verbatim}
CREATE USER kay IDENTIFIED BY mary DEFAULT TABLESPACE users QUOTA 1 M ON users;
GRANT connect TO kay;
GRANT CREATE TABLE TO kay;
GRANT SELECT ON scott.emp TO kay;
\end{verbatim}

\paragraph{Copie de la table \texttt{emp}}

\begin{verbatim}
CREATE TABLE emp AS ( SELECT * FROM scott.emp );
\end{verbatim}

\subsubsection{Informations sur les nouveaux utilisateurs} 

\begin{verbatim}
SELECT * FROM dba_users WHERE username = 'KAY' OR username = 'BOB';
\end{verbatim}

Le résultat est peu lisible, à cause du nombre cde colonnes de cette table.
Une sélection plus spécifique permettrait d’avoir des données manipulables.

\subsubsection{Quantité d’espace}

\begin{verbatim}
SELECT tablespace_name, SUM( bytes )
FROM dba_users, dba_free_space
WHERE username = 'BOB' AND tablespace_name = default_tablespace
GROUP BY tablespace_name;
\end{verbatim}

Résultat :
\begin{verbatim}
TABLESPACE_NAME                SUM(BYTES)                                       
------------------------------ ----------                                       
USERS                             1179648 
\end{verbatim}

\subsubsection{suppression du quota}

\begin{verbatim}
ALTER USER bob QUOTA 0 M ON users;
\end{verbatim}

\subsubsection{Suppression de Kay}

La suppression n’a pas pu être effectuée immédiatement, Kay étant connecté
dans un autre terminal. Nous avons donc respecté la procédure suivante :

\begin{verbatim}
SQL> SELECT sid,serial# FROM v$session WHERE username = 'KAY';

       SID    SERIAL#                                                           
---------- ----------                                                           
        29         93   

SQL> ALTER SYSTEM KILL SESSION '29,93';

Système modifié.

SQL> DROP USER kay;

Utilisateur supprimé.                                                        
\end{verbatim}

\subsubsection{Récupération du compte de Bob}

\begin{verbatim}
ALTER USER bob IDENTIFIED BY olink PASSWORD EXPIRE;
\end{verbatim}

Pour supprimer Bob, nous avons suivi la même procédure que pour Kay, excepté la dernière commande :
\begin{verbatim}
DROP USER bob CASCADE;
\end{verbatim}
