\section{Quatrième sujet}
\subsection{Gestion des utilisateurs}

\subsubsection{Création de Bob}

\begin{verbatim}
CREATE USER bob IDENTIFIED BY along DEFAULT TABLESPACE users QUOTA 1 M ON users;
GRANT connect TO bob;
GRANT CREATE TABLE TO bob;

\end{verbatim}

\subsubsection{Création de Kay et compie de \texttt{emp}}

\paragraph{Création de Kay}
\begin{verbatim}
CREATE USER kay IDENTIFIED BY mary DEFAULT TABLESPACE users QUOTA 1 M ON users;
GRANT connect TO kay;
GRANT CREATE TABLE TO kay;
GRANT SELECT ON scott.emp TO kay;
\end{verbatim}

\paragraph{Copie de la table \texttt{emp}}

\begin{verbatim}
CREATE TABLE emp AS ( SELECT * FROM scott.emp );
\end{verbatim}

\subsubsection{Informations sur les nouveaux utilisateurs} 

\begin{verbatim}
SELECT * FROM dba_users WHERE username = 'KAY' OR username = 'BOB';
\end{verbatim}

Le résultat est peu lisible, à cause du nombre cde colonnes de cette table.
Une sélection plus spécifique permettrait d’avoir des données manipulables.

\subsubsection{Quantité d’espace}

\begin{verbatim}
SELECT tablespace_name, SUM( bytes )
FROM dba_users, dba_free_space
WHERE username = 'BOB' AND tablespace_name = default_tablespace
GROUP BY tablespace_name;
\end{verbatim}

Résultat :
\begin{verbatim}
TABLESPACE_NAME                SUM(BYTES)                                       
------------------------------ ----------                                       
USERS                             1179648 
\end{verbatim}

\subsubsection{suppression du quota}

\begin{verbatim}
ALTER USER bob QUOTA 0 M ON users;
\end{verbatim}

\subsubsection{Suppression de Kay}

La suppression n’a pas pu être effectuée immédiatement, Kay étant connecté
dans un autre terminal. Nous avons donc respecté la procédure suivante :

\begin{verbatim}
SQL> SELECT sid,serial# FROM v$session WHERE username = 'KAY';

       SID    SERIAL#                                                           
---------- ----------                                                           
        29         93   

SQL> ALTER SYSTEM KILL SESSION '29,93';

Système modifié.

SQL> DROP USER kay;

Utilisateur supprimé.                                                        
\end{verbatim}

\subsubsection{Récupération du compte de Bob}

\begin{verbatim}
ALTER USER bob IDENTIFIED BY olink PASSWORD EXPIRE;
\end{verbatim}

Pour supprimer Bob, nous avons suivi la même procédure que pour Kay, excepté la dernière commande :
\begin{verbatim}
DROP USER bob CASCADE;
\end{verbatim}

\subsection{Gestion des profils}

Le compte de Bob a été recréé pour cette partie comme pour la partie précédente.

\subsubsection{Profil pour deux sessions simultanées}

\begin{verbatim}
CREATE PROFILE lowaccess LIMIT sessions_per_user 2 idle_time 1;
\end{verbatim}

\subsubsection{Autorisation de Bob pour deux sessions}

\begin{verbatim}
ALTER USER bob PROFILE lowaccess;
SELECT profile FROM dba_users WHERE username = 'BOB';
\end{verbatim}

Résultat:
\begin{verbatim}
PROFILE                                                                         
------------------------------                                                  
LOWACCESS                                                                       
\end{verbatim}

\subsubsection{Vérification de l’accès simultané}

Pour pouvoir nous connecter simultanément, nous avons dû activer la
fonctionnalité de limitation :
\begin{verbatim}
ALTER SYSTEM SET resource_limit = true;
\end{verbatim}

L’utilisateur a pu se connecter deux fois :
\begin{verbatim}
SQL> SELECT sid,serial# FROM v$session WHERE username = 'BOB';

       SID    SERIAL#                                                           
---------- ----------                                                           
        31        753                                                           
        34        780                                                           
\end{verbatim}

\subsection{Gestion des privilèges}

\subsubsection{Création de Kay}

\begin{verbatim}
CREATE USER kay IDENTIFIED BY mary DEFAULT TABLESPACE users QUOTA 1 m ON users;
GRANT CONNECT TO kay;
GRANT CREATE TABLE, CREATE VIEW TO kay;
\end{verbatim}

\subsubsection{Copie de \texttt{dept}}

Pour pouvoir copier la structure, nous avons dû donner le droit \verb|SELECT ON scott.emp| à Kay.

\begin{verbatim}
CONNECT kay/mary;
CREATE TABLE dept AS ( SELECT * FROM scott.dept WHERE 0 = 1 );
EXIT;
CONNECT / AS sysdba;
INSERT INTO kay.dept ( SELECT * FROM scott.dept );
GRANT SELECT ON kay.dept TO bob;
\end{verbatim}

\subsubsection{Autorisation avec droit d’accorder}

\begin{verbatim}
GRANT SELECT ON kay.dept TO bob WITH GRANT OPTION;
\end{verbatim}

\subsubsection{Création de Todd}

\begin{verbatim}
CREATE USER todd IDENTIFIED BY tr_uc DEFAULT TABLESPACE users QUOTA 1 M ON users;
GRANT connect TO todd;
\end{verbatim}

\subsubsection{Conportement de \texttt{WITH GRANT OPTION}}

\begin{verbatim}
CONNECT bob/along;
GRANT SELECT ON kay.dept TO todd;
EXIT;
CONNECT kay/mary;
REVOKE SELECT ON kay.dept FROM bob;
\end{verbatim}

Todd ne peut plus lire la table \texttt{dept} de Kay.

\subsubsection{Creation dans tous les schémas}

\begin{verbatim}
CONNECT / AS sysdba;
GRANT CREATE ANY TABLE TO kay;
EXIT;
CONNECT kay/mary;
CREATE TABLE bob.dept AS ( SELECT * FROM dept );
\end{verbatim}

La commande de création de table est acceptée et la table est créée :
\begin{verbatim}
SQL> select owner, table_name from dba_tables where table_name='DEPT';

OWNER                          TABLE_NAME                                       
------------------------------ ------------------------------                   
BOB                            DEPT                                             
KAY                            DEPT                                             
SCOTT                          DEPT                                             
\end{verbatim}
