\section{Troisième sujet}

\subsection{Gestion des fichiers de données et des tablespace}

\subsubsection{}

\subsection{Segments}

\subsubsection{les différents types de segments dans la base de données}

Commande :
\begin{verbatim}
SELECT DISTINCT segment_type FROM dba_segments;
\end{verbatim}

Résultat :
\begin{itemize}
\item \verb|LOB INDEX|
\item \verb|INDEX PARTITION|
\item \verb|NESTED TABLE|
\item \verb|TABLE PARTITION|
\item \verb|ROLLBACK|
\item \verb|LOB PARTITION|
\item \verb|LOB SEGMENT|
\item \verb|TABLE|
\item \verb|INDEX|
\item \verb|CLUSTER|
\item \verb|TYPE2 UNDO|
\end{itemize}

\subsubsection{Fichiers contenant l'espace alloué pour la table \texttt{emp} de Scott}

Commande :
\begin{verbatim}
SELECT file_name FROM dba_extents e, dba_data_files f WHERE owner = 'SCOTT' AND segment_name = 'EMP' AND e.file_id = f.file_id;
\end{verbatim}

\subsubsection{Espace libre par tablespace}

Commande :
\begin{verbatimtab}
SELECT	tablespace_name,
	COUNT(*) extents_count,
	SUM(bytes) "SIZE",
	MAX(bytes) max_extent_size
FROM dba_free_space
GROUP BY tabespace_name;
\end{verbatimtab}